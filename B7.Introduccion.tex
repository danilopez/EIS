\chapterbeginx{Introducción y visión general}
\minitoc

\begin{sinopsis}
\label{sec:intro:sinop}
	Introducción.

\end{sinopsis}

\sectionx{Objetivo}
\label{sec:intro:obj}
	En esta sección, se describe el \miindex{objetivo del proyecto}, es decir, qué pretende, a qué aspira, cuál es su meta.\nli
	Es importante comprender esta sección, porque de otro modo, no se entiende el resto de la documentación.

\sectionx{Estado del arte}
	Un proyecto se realiza sobre un \miindex{estado de la técnica} que debe explicarse para entender mejor conceptos tales como los problemas existentes o cuáles son las soluciones que se emplean hasta la fecha actual.

	El \miindex{estado del arte}, a veces llamado estado de la técnica, suele estar presente en este tipo de documentos.
  
\sectionx{Metodología y directrices seguidas}
	Durante la elaboración del proyecto, se siguen procedimientos que el lector necesita conocer para entender de forma integral todo el documento.

\sectionx{Estructura del documento}
 En esta sección, se explican los posteriores capítulos u otra información adicional que el proyecto contenga.

\sectionx{Ámbito de aplicación}
	Por último, completando los apartados anteriores, se explican las áreas de las que se compone el proyecto.

\chapterend