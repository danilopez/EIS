%%%%%%%%%%%%%%%%%%%%%%%%%%%%%%%%%%%%%%%%%%%%%%%%%%%%%%%%%%%%%%%%%%%
%%% Documento LaTeX 																						%%%
%%%%%%%%%%%%%%%%%%%%%%%%%%%%%%%%%%%%%%%%%%%%%%%%%%%%%%%%%%%%%%%%%%%
% T�tulo:	Par�metros de estilo
% Autor:  Ignacio Moreno Doblas
% Fecha:  2014-02-01
%%%%%%%%%%%%%%%%%%%%%%%%%%%%%%%%%%%%%%%%%%%%%%%%%%%%%%%%%%%%%%%%%%%
% Tabla de materias:
%--------------------%
% 1 M�rgenes de p�gina
%%%%%%%%%%%%%%%%%%%%%%%
% Para conocer los par�metros de dise�o de p�ginas, tales como
%  los m�rgenes izquierdo, derecho, anchura de p�gina, etc.
%  v�ase el archivo ``Page layout.png'' que acompa�a esta plantilla.
% As� se conocer� mejor c�mo adaptar el documento seg�n los 
%  requisitos del usuario.

% 1 M�rgenes de p�gina
%-------------------------------%
% Par�metros de estilo de p�gina.
% DIN A4: 29.7 cm x 21 cm
%		�rea neta: 3 cm + 3 cm + 15 cm.
%
% Definici�n de m�rgenes de p�gina
%  even para p�ginas pares
%  odd  para p�ginas impares
\newlength{\realoddsidemargin}	  % \oddsidemargin menos 1 in.
\newlength{\realevensidemargin}		% \evensidemargin menos 1 in.
\newlength{\realtopmargin}				% \topmargin menos 1 in.
%
% Asignaci�n de m�rgenes de p�gina
% ASIGNESE en caso de querer cambiarlo
\setlength{\realtopmargin}{1in}			% REAL top margin.
\setlength{\realoddsidemargin}{1in}		% REAL oddside margin.
\setlength{\realevensidemargin}{1in}	% REAL evenside margin.
\setlength{\hoffset}{0cm}
\setlength{\voffset}{0cm}
%
% Substracci�n de 1 pulgada de compensaci�n
%  (v�ase ``Page Layout.png'' para m�s informaci�n)
\addtolength{\realoddsidemargin}{-1in}	% 1 inch = 2.54 cm.
\addtolength{\realevensidemargin}{-1in}
\addtolength{\realtopmargin}{-1in}
%
% Asignaci�n de anchuras y m�rgenes
% No hay notas al margen
\setlength{\marginparsep}{0cm} % No van a existir notas al margen
\setlength{\marginparwidth}{0cm} % No van a existir notas al margen
%
% Asignaci�n de anchura de texto
\setlength{\textwidth}{15cm}	% Anchura neta del texto (globalmente).
%
% Asignaci�n de m�rgenes par, impar y en altura
\setlength{\oddsidemargin}{\realoddsidemargin}	% odd-page left margin (global).
\setlength{\evensidemargin}{\realoddsidemargin}	% even-page left margin (global).
\setlength{\topmargin}{\realtopmargin}					% top margin (Global).

\linespread{1.3}
% Se puede usar tambi�n el paquete chngpage.

\setlength{\headheight}{15pt}
%%%%%%%%%%%%%%%%%%%%%%%%%%%%%%%%%%%%%%%%%%%%%%%%%%%%%%%%%%%%%%%%%%
%		1 Length commands.					%
%-------------------------------%
% Defines new length command (e.g., \newlength{\gnat}}
%	\newlength{}
%
% Set lenght to a value.
%	\setlength{\gnat}{length}
%	\addtolength{}{}
%
% Sets the value of a length command equal to the width of a specified piece of text; e.g., \settowidth{\parindent}{\em small}.
%	\settowidth{}{}
% Set the value of a height. e.g., \settoheight{\parskip}{Gnu}
%	\settoheight{}{}
% Set the value that extends below the line. e.g., \settodepth{\parskip}{gnu}.
%	\settodepth{}{}
%
% To multiply a length, write: 7.0\gnat = \gnat * 7.0
%%%%%%%%%%%%%%%%%%%%%%%%%%%%%%%%%%%%%%%%%%%%%%%%%%%%%%%%%%%%%%%%%%