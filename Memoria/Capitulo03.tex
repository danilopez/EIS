%!TEX root = A0.Master.tex
\chapterbegin{Plan de pruebas}
En este capítulo se detallarán las pruebas realizadas a la aplicación desarrollada en este proyecto para verificar su correcto funcionamiento. Para efectuar dichas comprobaciones, se ha tomado un conjunto significativo de muestras, con el emulador incluido en el IDE y con dispositivos móviles reales.

\section{Pruebas de la aplicación móvil}
En esta sección se explican las pruebas a las que se ha sometido la aplicación cliente. En el desarrollo de la aplicación se ha pasado por dos fases. En la primera de ellas, la pantalla de reconocimiento de códigos de barras no estaba implementada y por ello se ha usado el emulador de Xcode. En la segunda, con el lector de códigos de barras ya en la aplicación, se han usado dispositivos móviles reales.

\subsection{Pruebas con el emulador}
\label{sec:emulador}
En este tipo de pruebas, realizadas durante el desarrollo de la aplicación, se ha comprobado principalmente que las llamadas al servidor fueran realizadas correctamente, y que la interfaz fuera diseñada de acorde a los requisitos.

Se han realizado las siguientes pruebas:

\begin{itemize}
	\item Comprobación de que la aplicación sea capaz de enviar mensajes al servidor.
	\item Comprobación de que los datos recibidos por el servidor sean mostrados en pantalla.
	\item Comprobación de que los elementos en pantalla son mostrados correctamente en distintos formatos de pantalla.
\end{itemize}

\subsection{Pruebas con dispositivos móviles reales}
Debido a las limitaciones del emulador, la mayoría de pruebas fueron realizadas en dispositivos móviles, donde se pudieron probar correctamente funciones tales como la cámara y la geolocalización.

Para las pruebas se han utilizado los terminales que aparecen en la Tabla \ref{tab:tabla-pruebas}, en la que se detalla también la versión de iOS instalada en el dispositivo y el tamaño en pulgadas de la pantalla.

% Table generated by Excel2LaTeX from sheet 'Sheet1'
\begin{table}[htbp]
  \centering
      \begin{tabular}{rrrr}
    \toprule
    Dispositivo & Versión iOS & Tamaño de pantalla & Resolución de pantalla \\
    \midrule
    iPhone 5 & 8.4.1 & 4"    & 640x1136 píxeles \\
    iPhone 5 & 9.0   & 4"    & 640x1136 píxeles \\
    iPhone 6 & 9.2   & 4,7"  & 750x1334 píxeles \\
    iPhone 6s & 9.2   & 4,7"  & 750x1334 píxeles \\
    \bottomrule
    \end{tabular}%
    \caption{Dispositivos móviles empleados en las pruebas}
  \label{tab:tabla-pruebas}%
\end{table}%

Además de las pruebas ya realizadas en el apartado \ref{sec:emulador} y que se necesitó validar en los dispositivos móviles reaes, se realizaron los siguientes tests:
\begin{itemize}
	\item Comprobación de que sólo se lee un código de barras formato \textit{Code 128}.
	\item Comprobación de que un código de barras en formato \textit{Code 128} es un código de producto.
	\item Comprobación de que el GPS del dispositivo móvil funciona correctamente, incluyendo la petición de permisos al usuario.
	\item Comprobación de cálculos de distancia a la tienda aún cuando la recepción del GPS del dispositivo no es óptima.
	\item Comprobación de que el mapa se muestran tanto el usuario en su ubicación actual como el de la tienda seleccionada, por muy lejos o cerca que se encuentre.
	\item Comprobación de que se no muestren datos en pantalla que no se encuentren la base de datos del servidor.
\end{itemize}

Además de estas pruebas, se han tenido en cuenta algunos tests adicionales a los ya descritos anteriormente, exclusivos para los dispositivos reales, que se enumeran a continuación:

\begin{itemize}
	\item Inicio satisfactorio de la aplicación.
	\item Comprobación de que los elementos de pantalla no puedan ser malipulados por el usuario.
	\item Comprobación de que se ejecutan correctamente la petición y recepción de datos del servidor.
	\item Comprobación de que todas las pantallas funcionan correctamente independientemente del tamaño y resolución de la pantalla del dispositivo.
\end{itemize}

\section{Análisis de los resultados}
Las pruebas realizadas sobre los dispositivos móviles reales constataron la importancia de disponer de varios modelos de dispositivos con distintos tamaños de pantalla y sistemas operativos.

Debido a la naturaleza de nuestra aplicación, desde muy al principio del proceso de desarrollo se hizo necesario el uso de un dispositivo, para poder comprobar el correcto funcionamiento de la cámara y la geolocalización.

A esto hay que añadir que el retardo introducido en la petición y recepción de datos entre el emulador y el servidor es prácticamente despreciable (el ordenador sobre el que se ejecuta el entorno de desarrollo está conectado a una línea de internet de fibra óptica), mientras que en el dispositivo móvil el tiempo de respuesta es variable en función de la capacidad de la red a la que se encuentre conectado y de la calidad de recepción de la señal móvil en caso de funcionar bajo red móvil. No obstante, este tiempo es lo suficientemente pequeño como para que no afecte a la navegación normal de la aplicación.

\chapterend