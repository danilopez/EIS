% Encabezamiento %
\pagestyle{fancy}
\fancyhead[LE,RO]{\thepage}
\fancyhead[LO]{Bibliografía}
%\fancyhead[RE]{Parte \thepart \rightmark} %
\fancyhead[RE]{\nouppercase{\rightmark}} %

%Inclusión de bibliografía%
\bibliography{E2.Bibliografia} %Úsese el nombre del fichero sin extensión

%Inclusión en el índice (Tabla de contenidos)
\addcontentsline{toc}{chapter}{Bibliografía}

%Formateo de estilo de bibliografía
% Otros formatos: plain, unsrt, abbrv
%  plain: las entradas se ordenan alfabéticamente y se etiquetan con un número (p.ej., [1])
% unsrt: igual que plain, pero aparecen en orden de citación.
% alpha: el etiquetado se hace por autor y año de publicación (p.ej., [Knu66]).
% abbrv: igual que alpha, pero más abreviado.
\bibliographystyle{alpha}

%Impresión de todas las entradas bibliográficas
\nocite{*}

\chapterend