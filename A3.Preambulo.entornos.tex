%%%%%%%%%%%%%%%%%%%%%%%%%%%%%%%%%%%%%%%%%%%%%%%%%%%%%%%%%%%%%%%%%%%
%%% Documento LaTeX 																						%%%
%%%%%%%%%%%%%%%%%%%%%%%%%%%%%%%%%%%%%%%%%%%%%%%%%%%%%%%%%%%%%%%%%%%
% T�tulo:	Entornos
% Autor:  Ignacio Moreno Doblas
% Fecha:  2014-02-01
%%%%%%%%%%%%%%%%%%%%%%%%%%%%%%%%%%%%%%%%%%%%%%%%%%%%%%%%%%%%%%%%%%%
% Tabla de materias:
%--------------------%
% 1 dobleindent
% 2 izqindent
% 3 dobleindentx
% 4 ite
% 5 descript
% 6 enu
% 7 itemization
% 8 sinopsis
%%%%%%%%%%%%%%%%%%%%%%%
% Para conocer los par�metros de dise�o de las listas, tales como
%  los m�rgenes izquierdo, derechos y los diferentes saltos,
%  v�ase el archivo ``List layout.png'' que acompa�a esta plantilla.
% As� se conocer� mejor c�mo adaptar un entorno seg�n los requisitos 
%  del usuario.

%%%%%%%%%%%%%%%%%%%%%%%
% Definici�n de longitudes para usar en los entornos:
%
% Normal parskip.
\newlength{\parskipenv}
\setlength{\parskipenv}{\parskip}

\newlength{\parindentenv}
\setlength{\parindentenv}{\parindent}
%%%%%%%%%%%%%%%%%%%%%%%

% 1 dobleindent
%El entorno dobleindent est� pensando para escribir p�rrafos con doble indentaci�n a cada lado.
%Tiene dos par�metros de entrada con las distancias medidas desde los m�rgenes de p�gina.

\newenvironment{dobleindent}[2]
	%Comienzo de nuevo entorno%
	{
	\begin{list}
		{}
		{
		% Left and right margins:
		\leftmargin = #1 
		\rightmargin = #2
		%
		% Separation from preceding and following text:
		\topsep = 0ex
		\partopsep = 0ex
		\parsep = \parskipenv
		%
		% Indentation for paragraphs:
		\itemsep = \parskipenv
		\itemindent = \parindentenv
		\listparindent = \itemindent
		%
		% Horizontal separation from label:
		\labelsep = 1ex
		\settowidth{\labelwidth}{0cm}
		}
		
		 \item}
	% End new env
	{\end{list}}

%%%%%%%%%%%%%%%%%%%%%%%%%%%%%
%2 izqindent
% El entorno izqindent s�lo crea un p�rrafo indentado a la izquierda.
\newenvironment{izqindent}[1]
{
\begin{dobleindent}{#1}{0cm}
}
{
\end{dobleindent}
}

%%%%%%%%%%%%%%%%%%%%%%%%%%%%%
% 3 dobleindentx
% El entorno dobleindentx es una variaci�n del dobleindent usando leftskip y rightskip.
% Aunque es m�s limitado, tambi�n se puede usar.
\newenvironment{dobleindentx}[2] % S�lo funciona en modo paragraph
{ % Preamble
  \leftskip = #1
  \rightskip = #2
}
{ % Postamble
\leftskip = 0cm
\rightskip = 0cm
}

%%%%%%%%%%%%%%%%%%%%%%%%%%%%%
% 4 ite
% El entorno ite es una modificaci�n del entorno itemize est�ndar de \LaTeX. Puede usarse o modificarse si el usuario lo desea.
% Tambi�n puede parametrizarse el entorno enumerate o description de forma equivalente.
\newenvironment{ite}
	{
		\begin{izqindent}{\parindent}
		\hspace{-\parindent} 	% compensaci�n del sangrado que introduce el entorno.
		\vspace{-1.0\parskip}	% compensaci�n del \parskip que introduce el entorno.
		\vspace{-\baselineskip}	% compensaci�n por la l�nea que introduce el entorno.
		\begin{itemize}
	}
	{
		\end{itemize}
		\end{izqindent}
	}

%%%%%%%%%%%%%%%%%%%%%5
% commando stdformat para formatear los entornos descript, enu y itemization.
\newcommand{\stdformat}
	{% Declarations for format presentation.
		%		  
		% Separation from preceding and following text:
		\setlength{\topsep}{0ex}%
		\setlength{\partopsep}{0ex} %
		%
		% Horizontal separation from label:
		\labelsep = 1ex
		\setlength{\labelwidth}{0ex}
		%
		%	Left and right margins:	
		\setlength{\leftmargin}{1cm}%
		\addtolength{\leftmargin}{\labelsep}
		\setlength{\rightmargin}{0ex}
		%  
		% Indentation for paragraphs:
		\setlength{\itemindent}{-\leftmargin}%
		\addtolength{\itemindent}{1ex}
		\setlength{\listparindent}{\parindent}%
		%		
		% Separation between paragraphs.
		\setlength{\parsep}{\parskipenv}% 
		\setlength{\itemsep}{1ex}
	}

%%%%%%%%%%%%%%%%%%%%%%%%%%%%%
% 5 descript

\newenvironment{descript}
	% Beginning new env def.
	{
		\begin{list}
			{} % No default label for \item.
			{
				% Declarations for format presentation.
				\stdformat
				%
				\renewcommand{\makelabel}[1]{\normalfont\bfseries##1\hfil}
			}
	}
	% Ending new env def.
	{
		\hspace*{\fill} \\ \end{list}
	} % Se introduce un salto de l�nea para que el texto siguiente est� separado.
%END newenvironment{descript}

%%%%%%%%%%%%%%%%%%%%%%%%%%%%%
% 6 enu
\newcounter{itemnumber} % Counter for the environment.

\newenvironment{enu}
	% Beginning new env def.
	{
		\begin{list}
		{
			\raggedleft \arabic{itemnumber}
		}
		{
			\usecounter{itemnumber}
			\stdformat
		}
	}
	{
		\end{list}
	}

%%%%%%%%%%%%%%%%%%%%%%%%%%%%%
% 7 itemization
\newenvironment{itemization}
	% Beginning new env def.
	{
		\begin{list}
			{$\bullet$} % No default label for \item.
			{
				% Declarations for format presentation.
				\stdformat
			}
	}
	% Ending new env def.
	{
		\end{list}
	}


%%%%%%%%%%%%%%%%%%%%%%%%%%%%%
% 8 sinopsis
\newenvironment{sinopsis}{%[1]{
	\sectiony{Sinopsis}
	%\label{#1}
} {
	\pagebreak
}